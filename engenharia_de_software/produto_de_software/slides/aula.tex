\documentclass[11pt]{beamer}
\mode<presentation>
\let\Tiny=\tiny
\usetheme{CambridgeUS}
\usefonttheme{professionalfonts}
\usepackage[brazil]{babel}
\usepackage[utf8]{inputenc}
\newtheorem{mydef}{Definição}
\newtheorem{myexample}{Exemplo}

\title{Produto de \textit{Software}}
\author{}
\date{}

\begin{document}

   \begin{frame}[plain]
        \titlepage
   \end{frame}

   \section{Introdução}

   \begin{frame}{Engenharia de de \textit{software}}
      \begin{itemize}
         \item A Engenharia de \textit{Software} surgiu nos anos 70 para transpor os processos de negócio em \textit{software} customizável.
         \item Os requisitos vinham dos clientes e eram refinados junto à equipe de desenvolvimento.
      \end{itemize}
   \end{frame}

   \begin{frame}{Engenharia de \textit{software} ontem e hoje}
      \begin{itemize}
         \item O foco no projeto de \textit{software} dominou a indústria por mais de duas décadas e moldou todo conjunto de conhecimentos que compõem a Engenharia de \textit{Software}.
         \item Quanto mais os processos de negócio se desenvolveram, mais notável se tornou a percepção que muitos negócios não precisavam de \textit{software} customizado.
         \item Produtos genéricos de \textit{software} podem muito bem atender a diversos problemas de negócio simples.
      \end{itemize}
   \end{frame}
   
   \begin{frame}{Produto de \textit{software}}
      \begin{mydef}
         "Produtos de \textit{software} são sistemas computacionais genéricos vendidos a governos, empresas e consumidores." (Sommerville, 2020, trad. nossa)
      \end{mydef}
   \end{frame}

   \begin{frame}{Produto de \textit{software}}
      Desenhados para:
      \begin{itemize}
         \item suporte a processos de negócio;
         \item ferramentas de produtividade;
         \item jogos;
         \item sistemas de informação pessoal.
      \end{itemize}
   \end{frame}

   \section{Projeto de \textit{software} $\times$ Produto de \textit{software}}

   \begin{frame}{Projeto de \textit{software} $\times$ Produto de \textit{software}}
      \begin{itemize}
         \item Um das diferenças maiores entre projeto e produto de \textit{software} se dá nos requisitos.
         \item No projeto de \textit{software}, o cliente define os requisitos e paga por eles.
         \item Se o processo do cliente muda, então mudam os requisitos.
         \item No produto de \textit{software}, quem define os requisitos é o desenvolvedor (empresa) baseado na observação dos problemas de potenciais clientes.      
      \end{itemize}
   \end{frame}

   \begin{frame}{Projeto de \textit{software} $\times$ Produto de \textit{software}}
      \begin{itemize}
         \item O custo do produto de \textit{software} tende a ser mais baixo para o cliente, pois o custo de produção é dividido entre mais clientes.
         \item Sommerville afirma que o controle do produto de \textit{software} pelo desenvolvedor gera mais riscos para o cliente.
         \item Isso é discutível. Dependerá do contrato.
      \end{itemize}
   \end{frame}

   \begin{frame}{Projeto de \textit{software} $\times$ Produto de \textit{software}}
      \begin{itemize}
         \item Sommerville também coloca que produtos excelentes comumente falham porque outros produtos inferiores chegam ao mercado primeiro.
         \item Isso é também discutível. Depende de diversos fatores, como a expansão do mercado.
         \item Por exemplo, a Amazon é a pioneira em \textit{Cloud Computing}, porém Google, Microsoft, Alibaba, IBM e outras conseguem algum nível de competição.
         \item Outro exemplo foram as pioneiras Microsoft e Symbian nos \textit{smartphones}, mas foram logo ultrapassadas por Blackberry, Apple e demais.
      \end{itemize}
   \end{frame}

   \section{Visão do produto}

   \begin{frame}{Visão do produto}
      \begin{itemize}
         \item Segundo Sommerville, o ponto de partida para o desenvolvimento de um produto deveria ser a \textbf{visão do produto}.
         \item Esta é a descrição sucinta e simples do produto a ser desenvolvido e o seu diferencial perante os concorrentes.
         \item Todas funcionalidades e características do \textit{software} devem estar de acordo com a visão do produto.
      \end{itemize}
   \end{frame}

   \begin{frame}{Visão do produto}
      Sommerville define três questionamentos que embasam a criação da visão do produto, a saber:
      \begin{itemize}
         \item \textbf{O que} é o produto a ser desenvolvido? \textbf{O que} faz desse produto diferente dos seus competidores?
         \item \textbf{Quem} é o público-alvo deste produto?
         \item \textbf{Por que} clientes compram esse produto?
      \end{itemize}
   \end{frame}

   \begin{frame}{Visão do produto}
      No livro Crossing the Chasm, Geoffrey Moore sugere a seguinte estruturação para escrever a visão do produto baseada em palavras-chave:
      \begin{itemize}
         \item PARA (cliente alvo);
         \item QUE (necessidade ou oportunidade);
         \item O (nome do produto) É UM (categoria do produto);
         \item QUE (benefício-chave ou razão para comprar);
         \item DIFERENTE DE (competidor);
         \item NOSSO PRODUTO (diferenciação).
      \end{itemize}
   \end{frame}

   \begin{frame}{Visão do produto}
      \textbf{PARA} estudantes surdos e professores, \textbf{QUE} necessitam de um portal educativo com \textit{interface} adaptada para o público surdo, \textbf{O} Portal Mão Amiga \textbf{É UM} \textit{software} gerenciador de conteúdo educacional \textbf{QUE} possui \textit{interace} completamente adaptada às necessidades dos alunos surdos. \textbf{DIFERENTE DO} Moodle e similares, \textbf{NOSSO PRODUTO} permite que o surdo acesse os conteúdos sem auxílio de intérprete ou treinamento.
   \end{frame}

   \section{Prototipação}

   \begin{frame}{Prototipação}
      \begin{itemize}
         \item Os produtos de \textit{software} surgem geralmente da identificação de oportunidades.
         \item É sugerida criação de protótipo como resultado de uma primeira versão.
         \item Um protótipo é uma ferramenta muito boa tanto para provar que a proposta de do produto é factível quanto para atrair investidores.
      \end{itemize}
   \end{frame}

   \begin{frame}{Prototipação}
      \begin{itemize}
         \item Para \textit{softwares} de uso interno, como os utilizados por grupos de pesquisa, um protótipo pode ser o bastante.
         \item Quando este \textit{software} for utilizado externamente, recomenda-se "jogá-lo fora" e refazê-lo seguindos boas práticas e parâmetros de confiança e segurança.
      \end{itemize}
   \end{frame}

   \begin{frame}{Referências}
      \begin{itemize}
          \item Sommerville, Ian. Software Engineering - Global Edition. 10ed. 2016. Pearson Education.
          \item Sommerville, Ian. Engineering Software Products: An Introduction to Modern Software Engineering. 1ed. 2021. Pearson Education. 
      \end{itemize}
   \end{frame}

\end{document}